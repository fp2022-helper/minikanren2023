% !TEX TS-program = xelatex
% !TeX spellcheck = ru_RU

\documentclass{article}

\usepackage{hyperref}
\usepackage{polyglossia}
\setmainlanguage{russian}
\setotherlanguage{english}
\usepackage{csquotes}
\usepackage{multirow}
\pagenumbering{gobble} 
\usepackage[a5paper]{geometry}
\geometry
  {top=18pt
  , bottom=14pt
  ,left=.5cm,right=1cm
%  ,paperwidth=5.5in
%  ,paperheight=8.5in,
  }

\setmainfont[
 Ligatures=TeX,
 Extension=.otf,
 BoldFont=cmunbx,
 ItalicFont=cmunti,
 BoldItalicFont=cmunbi,
]{cmunrm}
% С засечками (для заголовков)
\setsansfont[
 Ligatures=TeX,
 Extension=.otf,
 BoldFont=cmunsx,
 ItalicFont=cmunsi,
]{cmunss}
\setmonofont[Scale=1.0,
    BoldFont=lmmonolt10-bold.otf,
    ItalicFont=lmmono10-italic.otf,
    BoldItalicFont=lmmonoproplt10-boldoblique.otf
]{lmmono9-regular.otf}

%\renewcommand{\epsilon}{\ensuremath{\varepsilon}}
\usepackage{graphicx}

\title{Отзыв научного руководителя о прохождении
  учебной практики \\
   ``Интеграция Qt/QML c OCaml: вызов обработчиков''}

\date{\today}
%\author{Косарев Дмитрий}
\def\Jsoo{\textsc{Js\_of\_ocaml}}
\def\Javascript{\textsc{Javascript}}
\def\OCaml{\textsc{OCaml}}

\begin{document}
\maketitle

\begin{tabular}{l*{6}{|c} r}
Team              & \multicolumn{3}{ |c| }{Kotlin}  & \multicolumn{3}{ |c| }{Racket}   & Pts \\
\hline
 
        & u & ms & ms/1ku  & u & ms & ms/1ku & ?  \\

$3^5$ 
        & 823212 & 4 &   & 342799 & 10 & 5 & 12  \\
$logo_3 243$           
        & 74042 & 2 &   & 44410 &  7 & 8 &  7  \\
$127\cdot127$            
        & 220986 & 3 &   & 133947 &  8 & 9 &  9  \\
$7^2$    
        & 6 & 2 &   & 299688  &   & 8 &  7  \\
\end{tabular}
  
\vspace{3em}
\begin{tabular}{l*{6}{ |c }}
\multirow{2}{*}{} & \multicolumn{3}{ |c| }{Kotlin}  & \multicolumn{3}{ |c| }{Racket}  \\
\cline{2-7}
 & unifications & ms & U/ms  & unifications & ms & U/ms   \\
\hline
         $3^5$ &    342799 & 2118.40 &   162 &    342799 & 137.64 &  2491    \\
   $log_3 243$ &     74042 & 3161.08 &    23 &    44410 & 19.83 &  2239    \\
$127\cdot 127$ &    220986 & 406.59 &   544 &    133947 & 72.11 &  1857    \\
$225\cdot 225$ &    894219 & 1876.77 &   476 &    543997 & 339.58 &  1602    \\
         $7^2$ &    385752 & 887.93 &   434 &    299688 & 105.97 &  2828    \\
\end{tabular}

\end{document}